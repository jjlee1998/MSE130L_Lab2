1/8" diameter rods of 304SS and 1018MS were cut into cylindrical samples, then sanded with 600-grit paper and rinsed with deionized water so as to clean their surfaces.  In total, four samples were produced (2x 304SS and 2x 1018MS).  The samples were imaged under an optical microscope so as to provide a reference against which to compare their corroded surfaces.

Each sample was installed as the working electrode within a polarization cell, alongside a platinum counter-electrode and a saturated calomel reference electrode (SCE).  So as to minimize the effects of solution resistivity on the measured voltage, the reference electrode was contained within a Lugger-Habin probe, of which the capillary tip was placed at the midpoint of the sample.  The polarization cell was filled with either 250 mL of 1M HCl solution or 250 mL of 1M H2SO4 solution.  The Luggin-Haber probe was filled with the same chosen solution to a surface level 1/4" below that of the polarization cell; this precluded the contamination of the polarization cell with 4M Cl- solution from the SCE reference.  Care was taken during filling to avoid bubbles, as well as to position the Luggin-Haber probe in a way that would not allow bubbles to enter its capillary during the course of the experiment.  Sample dimensions, solutions, and immersion lengths are recorded in Table \ref{table:samples}.

\begin{table}[h!]
	\centering
	\begin{tabular}{lcccc}
	\toprule
	Sample & Materials & Soln & Diameter & Immersed Length \\
	\midrule
	1 & 1018MS & H$_2$SO$_4$ & 3.12 mm & 14.8 mm \\
	2 & 1018MS & HCl	 & 3.12 mm & 15.6 mm \\
	3 & 304SS  & H$_2$SO$_4$ & 3.12 mm & 16.1 mm \\
	4 & 304SS  & HCl	 & 3.12 mm & 13.6 mm \\
	\bottomrule
	\end{tabular}
	\caption{Sample dimensions used to calculate critical current densities from the collected current data.}
	\label{table:samples}
\end{table}

The three electrodes were connected to a potentiostat constructed by previous laboratory personnel (10V compliance voltage, 190 mA maximum current, 200 mA thermal fuse cutoff, 15pA reference input current, 1e-6 mA current sensitivity, 2\% current accuracy).  The approximate location of the H/Fe corrosion potential was identified, after which various polarization sweeps were conducted for each sample.  Twelve polarization sweeps were collected in total.  The overall classification of each sweep are enumerated in Table \ref{table:sweeps}, while the parameters governing each sweep are contained within the expanded Table \ref{table:sweeps_full} in Appendix 2.  After the sweeps were concluded, the samples were dried and imaged once more under an optical microscope at magnifications of 5x, 10x, and 20x.

\begin{table}[h!]
	\centering
	\begin{tabular}{lccp{2.5cm}p{2.5cm}p{6cm}}
	\toprule
	Scan & Sample & Solution & Classification & Sweep Rate (mV/sec) \\
	\midrule
	1 & 1018MS & H2SO4 & Ano/Cat & 1 mV/sec \\
	2 & 1018MS & H2SO4 & LPR & 0.1 mV/sec \\
	3 & 1018MS & H2SO4 & Ano/Cat & 1 mV/sec  \\
	4 & 1018MS & H2SO4 & LPR & 0.1 mV/sec  \\
	5 & 1018MS & HCl & Ano/Cat & 1 mV/sec  \\
	6 & 1018MS & HCl & LPR & 0.1 mV/sec  \\
	7 & 1018MS & HCl & Ano/Cat & 1 mV/sec  \\
	8 & 1018MS & HCl & LPR & 0.1 mV/sec \\
	9 & 304SS & H2SO4 & Cathodic & 1mV/sec \\
	10 & 304SS & H2SO4 & Anodic & 1 mV/sec  \\
	11 & 304SS & HCl & Cathodic & 1 mV/sec  \\
	12 & 304SS & HCl & Anodic & 1 mV/sec  \\
	\bottomrule
	\end{tabular}
	\caption{General description of the 12 collected polarization curves.  For the detailed parameters governing their collection, see Table \ref{table:sweeps_full} in Appendix 2.}
	\label{table:sweeps}
\end{table}

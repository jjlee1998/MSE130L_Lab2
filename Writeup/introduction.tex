- the purpose of this report is to quantify the corrosion behavior of rod samples of 1018 mild carbon steel (1018MS +composition) and 304 stainless steel (304SS + composition) in strongly acidic solutions (1M HCl and 1M H2SO4), as submitted by an anonymous manufacturer for quality-control purposes.

- discussion of what happens when a metal is placed in solution with a counter-electrode to maintain neutrality - dissolution equilbrium potential (equal currents) difference with solution wrt the solution potential (encoded as voltage wrt the reference electrode) as given by the butler-volmer equation, with the cofactor given by <eqn> representing the oft-cited metric of the Tafel Slope, which may be understood as the relationship between overpotential (eta) and the resulting ln(i); and the intersection (zero current) representing the equilibrium potential that develops

- this behavior is modified by additional redox half-reactions that may now take place involving species in solution, which makes possible coupling between redox half-reactions involving different species

- in the case of mild steels in acidic solution, we are particularly interested in the coupling between the H+ reduction reaction and the Fe oxidation reaction (C in alpha-phase is negligible); in this case and in others when a metal dissolves into solution by coupling with a solution species, the resulting equilibrium potential is often referred to as the corrosion potential (m del s phi); potentials lower than the corrosion potential are called the "cathodic" regime because current must be supplied to the reaction, while potentials higher than the corrsion potential are the "anodic" regime, and the whole sweep is called the polarization curve of the material

- in the local vicinity (colloqually considered as +- 100 mV) of the corrosion potential, behavior is expected to closely match that predicted by the bulter-volmer equation (i.e. linear behavior in lin-lin space, from which the corrosion potential may be obtained by comparing the slope with BV)

- at larger potentials, the half-reaction may become diffusion limited (for example, by the formation of a polarized double-layer on the electrode surface); in these cases, the manifested current is depressed relative to the butler-volmer prediction; we may rudimentarily model this scenario by modifying the relevant butler-volmer term with a resistive component, producing (modified eqn)

- 1018MS is expected to further deviate from the behavior predicted by (bv and mod-bv) by virtue of its microstructure; 0.18 wt% C means X wt% Fe3C and (1-X) wt% alpha, which by virtue of cooling through the alpha+gamma sub-eutectoid region will manifest as Y wt% ferritic alpha-phase and (1-Y) wt% pearlite of composition #### -- Fe3C does not dissolve as the alpha-phase does, yet conducts metallically and catalyzes the reactions of the H/H+ redox couple by means of offering a much higher exchange current density (i0 in the bulter-volmer eqn) than on ferrite; this means that as the corrosion progresses and the Fe3C cementite lamellae are "excavated," we might expect the H+ reduction reaction to increase in rate

- 304 SS is expected to further deviate from the behavior predicted by (bv and mod-bv) by virtue of its engineered passivation behavior; namely, at a second threshold potential above the H/Fe corrosion potential, the Cr content oxidizes to form a "passivation layer" consisting largely of Cr2O3 (but other compounds too); this passivation layer may be considered as an extremely effective diffusion barrier, i.e. one in which the resistive terms of eqn (mod-bv) is enormous, producing an ohmic I/V response that is nearly flat

- at the even-higher "breakdown potential," the passivation layer may oxidize (eqn given in text), restoring the diffusion-limited Fe oxidation behavior after fully ablating; the breakdown potential may be slightly differentiated from the equilbrium potential of the Cr III/Cr VI redox couple

- at the even-even-higher water breakdown potential, most of the current response comes from the decomposition of H2O into oxygen gas (which is relatively insoluble in these acidic solutions), hydrogen ions, and electrons

- 304SS also manifests a particular behavior in HCl: chloride ions attack the passivation later in a self-catalyzed reaction, eating through the layer in local regions and manifesting as a return to diffusion-limited Fe oxidation behavior; this occurs at a lower potential than the breakdown potential and is called the "pitting potential"

- therefore in order to characterize the manufacturer's submitted samples, the goal is to quantify the following:
	1. 304SS: corrosion poten, local H/Fe tafel slopes, passivation poten, pitting poten in HCl, breakdown poten, Cr equilibrium poten, water breakdown poten
	2. 1018MS: corrosion poten, local H/Fe tafel slopes, effect of cementite microstructure evolution

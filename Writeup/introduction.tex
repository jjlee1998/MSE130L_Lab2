The purpose of this report is to quantify the corrosion behavior of rod samples of 1018 mild carbon steel ("1018MS") and 304 stainless steel ("304SS") when polarized in strongly acidic solutions.  In particular, the manufacturer has requested characterization in 1M HCl and 1M H$_2$SO$_4$, both of which have a pH of \textapprox 0, but which manifest different passivation behaviors at higher potentials due to their anion species.

\subsection{Three Models for Polarization Curve Fitting}

When a metal is placed in solution, there often exists a difference between the metal's work function and the electron energy levels of the solution species.  Electrons will therefore transfer between the metal and the solution. Depending on the properties of the resulting metal ions, this may result in the corrosion (dissolution) of of the metal body.  For example, for the coupled reactions in Equation \ref{eqn:couple_FeH}, the Fe$^{2+}$ ions produced are soluble in aqueous solution:
%
	\begin{equation}
	\begin{split}
		\text{Fe}_{(s)} \rightarrow \text{Fe}_{(aq)}^{2+} + 2 e^- \\
		2 \text{H}^+_{(aq)} + 2 e^- \rightarrow \text{H}_{2(g)} \\
	\end{split}
	\label{eqn:couple_FeH}
	\end{equation}
%

In an isolated system, this redox reaction continues until a the unfavorable charge imbalance cancels the driving force; an equilibrium electrochemical potential difference is thus established.  Yet the scenario is modified if this metal electrode is connected to a counter-electrode of a different metallic species.  The counter-electrode may manifest its own redox reactions so as to maintain charge neutrality in the solution; furthermore, any electrochemical potential difference between the electrode and counter-electrode provides a driving force for current flow.  The net effect is that a circuit is established that continuously corrodes the electrode.

This scenario is common in engineering situations -- for example, the field of marine electronics deals heavily with the fact that the ocean-immersed components of a ship's hull will corrode if made of dissimilar metals.  In the laboratory, it may be replicated within a polarization cell, wherein a sample is immersed in solution with a counter-electrode (typically Pt, which does not corrode) and a reference electrode against which to measure potential differences (such as the Saturated Calomel Electrode, a Hg-based electrode abbreviated as "SCE").  A potentiostat is used to drive the sample to various potentials relative to SCE; the current required to do so provides a measure of the reactions taking place at the sample surface.  Such data sweeps are known as polarization curves.

The corrosion behavior of Fe in acidic solution may be described using the Butler-Volmer equation:
%
	\begin{equation}
		j = j_{\text{corr}} \left[
		\exp\left(\frac{\beta n F}{RT}(\phi - \Delta\phi_{\text{corr}})\right)
		-\exp\left(\frac{(1-\beta) n F}{RT}(\phi - \Delta\phi_{\text{corr}})\right)
		\right]
	\label{eqn:bv_raw}
	\end{equation}
%
In this form of the Bulter-Volmer equation, $\phi$ is the applied potential relative to SHE.  $\Delta \phi_{\text{corr}}$ is the "corrosion potential," at which the rates of Fe oxidation and H reduction at the electrode are equal to each other.  The reaction rate at this point is the corrosion rate $j_{\text{corr}}$ and manifests as a point where zero current need be supplied by the potentiostat.  $\beta$ encodes any anisotropy between the exchange current densities of the anodic and cathodic reactions and is typically \textapprox 0.5; $n = 2$ is the number of electrons exchanged in the reaction; $R$ is the ideal gas constant; $F$ is the Faraday constant; and $T$ is temperature.

In this analysis, the Bulter-Volmer equation will not be used in its full form, but instead in three separate modified forms.  The first modification is to condense and rewrite the prefactors:
%
	\begin{equation}
		j = j_{\text{corr}} \left[
		\exp\left(\frac{\ln(10)\eta}{A_{\text{Fe}}}\right)
		-\exp\left(\frac{\ln(10)\eta}{A_{\text{H}}}\right)
		\right]
	\label{eqn:bv_tafel}
	\end{equation}
%
In this case, $\eta$ is shorthand for the overpotential $\phi - \Delta\phi_{\text{corr}}$.  $A_{\text{Fe}}$ and $A_{\text{H}}$ are the Tafel slopes for Fe oxidation and H reduction, respectively, and may be considered as the overpotential required (in volts) to increase the reaction current density by a factor of 10.  Fitting \ref{eqn:bv_tafel} to the polarization curve is typically done in the space of $\log_{10}|j|$ vs. $\eta$, such that either the anodic or cathodic terms dominate and produce linear behavior far from $\Delta \phi_{corr}$, while a singularity exists at $\Delta \phi_{corr}$ itself.

The second modification is to consider the Bulter-Volmer equation in the small-$\eta$ regime, i.e. at potentials close to $\Delta \phi_{\text{corr}}$ vs. SHE.  A Taylor expansion of Equation \ref{eqn:bv_raw} reveals linear behavior:
%
	\begin{equation}
		j = j_{\text{corr}}\frac{nF}{RT}\eta
	\label{eqn:bv_lpr}
	\end{equation}
%
For the Fe/H corrosion couple, the $nF/RT$ prefactor has a value of 77.85 V$^{-1}$ at 25\textdegree{C}. The corrosion potential may therefore be determined by a linear fit to a "Linear Polarization Resistance" (LPR) scan at small overpotentials (colloqually defined as $|\eta| < 100$ mV).

The third modification is to account for the limitations of ion diffusion: at large overpotentials, a diffusion barrier may slow the rate of reaction, producing a smaller current than would be expected from Equation \ref{eqn:bv_raw}.  Such diffusion barriers comprise complicated effects such as charge double-layers and may manifest themselves nonlinearly due to reaction kinetics, but as a rudimentary approximation, the barrier is assumed to have a constant resistivity $\rho_{\text{bar}}$.  As such, a given half-reaction in Equation \ref{eqn:bv_raw} will tend towards ohmic behavior at high currents, while Equation \ref{eqn:bv_raw} itself describes electrical behavior comparable to a diode.  The model in Reference \cite{} for a diode and resistor in series may be used as a guide to derive:
%
	\begin{equation}
		j = \frac{1}{B_0 \rho_{\text{bar}}}
		W(B_0 j_0 \rho_{\text{bar}} \exp(B_0 \eta))
	\label{eqn:bv_w}
	\end{equation}
%
In this equation, $B_0$ is defined as $ln(10) / A_0$ where $A_0$ is the Tafel-slope of the half-reaction, while $W_0$ is the 0-th branch of the Lambert W function.  Since this equation is presented for a half-reaction, the parameters are defined in terms of a reference potential $\Delta \phi_{\text{ref}}$ and reference current $\j_0$ instead of a corrosion potential and current, which only have meaning for a reaction couple.

\subsection{Additional Behavior: 1018MS}

1018MS (0.15-0.20 wt\% C, 0.60-0.90 wt\% Mn, bal. Fe) possesses a heterogenous microstructure consisting of a nearly-carbonless ferrite $\alpha$-phase and the high-carbon Fe$_3$C cementite phase.  The fabrication of this steel involves quenching through a two-phase $\alpha + \gamma$ region; as a result, the original volumes of $\gamma$-phase become "pearlite," comprising dense lamellae of ferrite and cementite.  An inspection of the Fe phase diagram for 0.18 wt\% C indicates that 1018MS globally contains 2.7 wt\% cementite, concentrated within pearlite regions that comprise 20 wt\% of the microstructure.  Cementite therefore comprises 14 wt\% of the pearlite regions.  For this analysis, the alloying element Mn is ignored.

This microstructure is generally expected to have a pronounced effect on the corrosion behavior of 1018MS.  Cementite does not corrode as the high-Fe $\alpha$-phase does, yet it comducts metallically and catalyzes the H$^+$ reduction reaction.  Specifically, the exchange current density on Fe$_3$C for hydrogen reduction is several orders of magnitude higher than on the $\alpha$-phase.  As the $\alpha$-phase corrodes away, Fe$_3$C lamellae are left behind, producing a large increase in the area available for Fe$_3$C to catalyze the reaction.  This means that as the 1018MS electrode corrodes, the parameters of Equation \ref{eqn:bv_tafel} should shift towards greater corrosion currents in the cathodic regime.

\subsection{Additional Behavior: 304SS}

304SS (18 wt\% Cr, 8 wt\% Ni, bal. Fe) is designed to self-passivate in acidic solution above certain overpotentials.  Specifically, at a passivation potential $\Delta \phi_{\text{pass}} > \Delta \phi_{\text{corr}}$, Cr atoms oxides into a +3 state in the form of a "passivation layer" consisting primarily of Cr$_2$O$_3$.  Despite being only a few unit cells thick, this passivation layer provides an extremely-effective barrier against continued Fe oxidation; above the $\Delta \phi_{\text{pass}}$, the electrode current in a passivation sweep is nearly flat.

This passivated regime, however, has an upper limit resulting from a "breakdown potential" ($\Delta \phi_{\text{breakdown}}$) at which Cr$_2$O$_3$ begins to oxidize into aqueous HCrO$_4^-$ ions (+6 Cr oxidation state).  In the presence of chloride ions, the limit is instead dictated by "pitting potential" ($\Delta \phi_{\text{pit}}$), wherein Cl$^-$ ions locally attack the passivation barrier in a self-catalyzed reaction, thus locally restoring Fe corrosion behavior.  This pitting potential typically occurs before the passivation layer breakdown potential; as such, the passivation regime in HCl is expected to be smaller than in H$_2$SO$_4$.

\subsection{Charactization Targets}

In light of the above discussion, the following quantities are desired in order to comprehensively characterize the samples from the manufacturer:

\begin{itemize}

\item 1018MS: $\Delta \phi_{\text{corr}}$, $A_{\text{Fe}}$, $A_{\text{H}}$, and effect of Fe$_3$C microstructural evolution.

\item 304SS: $\Delta \phi_{\text{pass}}$, $\Delta \phi_{\text{breakdown}}$, $\Delta \phi_{\text{pit}}$, and any anomalous polarization behavior.

\end{itemize}

\iffalse
- the purpose of this report is to quantify the corrosion behavior of rod samples of 1018 mild carbon steel (1018MS +composition) and 304 stainless steel (304SS + composition) in strongly acidic solutions (1M HCl and 1M H2SO4), as submitted by an anonymous manufacturer for quality-control purposes.

- discussion of what happens when a metal is placed in solution with a counter-electrode to maintain neutrality - dissolution equilbrium potential (equal currents) difference with solution wrt the solution potential (encoded as voltage wrt the reference electrode) as given by the butler-volmer equation, with the cofactor given by <eqn> representing the oft-cited metric of the Tafel Slope, which may be understood as the relationship between overpotential (eta) and the resulting ln(i); and the intersection (zero current) representing the equilibrium potential that develops

- this behavior is modified by additional redox half-reactions that may now take place involving species in solution, which makes possible coupling between redox half-reactions involving different species

- in the case of mild steels in acidic solution, we are particularly interested in the coupling between the H+ reduction reaction and the Fe oxidation reaction (C in alpha-phase is negligible); in this case and in others when a metal dissolves into solution by coupling with a solution species, the resulting equilibrium potential is often referred to as the corrosion potential (m del s phi); potentials lower than the corrosion potential are called the "cathodic" regime because current must be supplied to the reaction, while potentials higher than the corrsion potential are the "anodic" regime, and the whole sweep is called the polarization curve of the material

- in the local vicinity (colloqually considered as +- 100 mV) of the corrosion potential, behavior is expected to closely match that predicted by the bulter-volmer equation (i.e. linear behavior in lin-lin space, from which the corrosion potential may be obtained by comparing the slope with BV)

- at larger potentials, the half-reaction may become diffusion limited (for example, by the formation of a polarized double-layer on the electrode surface); in these cases, the manifested current is depressed relative to the butler-volmer prediction; we may rudimentarily model this scenario by modifying the relevant butler-volmer term with a resistive component, producing (modified eqn)

- 1018MS is expected to further deviate from the behavior predicted by (bv and mod-bv) by virtue of its microstructure; 0.18 wt% C means X wt% Fe3C and (1-X) wt% alpha, which by virtue of cooling through the alpha+gamma sub-eutectoid region will manifest as Y wt% ferritic alpha-phase and (1-Y) wt% pearlite of composition #### -- Fe3C does not dissolve as the alpha-phase does, yet conducts metallically and catalyzes the reactions of the H/H+ redox couple by means of offering a much higher exchange current density (i0 in the bulter-volmer eqn) than on ferrite; this means that as the corrosion progresses and the Fe3C cementite lamellae are "excavated," we might expect the H+ reduction reaction to increase in rate

- 304 SS is expected to further deviate from the behavior predicted by (bv and mod-bv) by virtue of its engineered passivation behavior; namely, at a second threshold potential above the H/Fe corrosion potential, the Cr content oxidizes to form a "passivation layer" consisting largely of Cr2O3 (but other compounds too); this passivation layer may be considered as an extremely effective diffusion barrier, i.e. one in which the resistive terms of eqn (mod-bv) is enormous, producing an ohmic I/V response that is nearly flat

- at the even-higher "breakdown potential," the passivation layer may oxidize (eqn given in text), restoring the diffusion-limited Fe oxidation behavior after fully ablating; the breakdown potential may be slightly differentiated from the equilbrium potential of the Cr III/Cr VI redox couple

- at the even-even-higher water breakdown potential, most of the current response comes from the decomposition of H2O into oxygen gas (which is relatively insoluble in these acidic solutions), hydrogen ions, and electrons

- 304SS also manifests a particular behavior in HCl: chloride ions attack the passivation later in a self-catalyzed reaction, eating through the layer in local regions and manifesting as a return to diffusion-limited Fe oxidation behavior; this occurs at a lower potential than the breakdown potential and is called the "pitting potential"

- therefore in order to characterize the manufacturer's submitted samples, the goal is to quantify the following:
	1. 304SS: corrosion poten, local H/Fe tafel slopes, passivation poten, pitting poten in HCl, breakdown poten, Cr equilibrium poten, water breakdown poten
	2. 1018MS: corrosion poten, local H/Fe tafel slopes, effect of cementite microstructure evolution
\fi

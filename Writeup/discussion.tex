\subsection{Comments Upon Fitting Procedure Validity}

In the above analyses, three different fitting procedures were deployed: (1) a Butler-Volmer equation using Tafel slopes, as described by Equation \ref{eqn:bv_tafel}; (2) a linearized Butler-Volmer equation applied in the low-$\eta$ limit following Equation \ref{eqn:bv_lpr}; and (3) a generalized diffusion and passivation half-reaction model described by Equations \ref{eqn:bv_w} and \ref{eqn:rho_jmak}.  Of the three, only the linearized model appears to produce consistent and reliable results, as qualified by the low variances of its fitted parameters.

The Tafel-slope equation produced consistent results across multiple scans, but a possible fitting artifact produced variances that were several orders of magnitude larger than the parameters themselves.  This might be attributed to the fact that variances lose meaning when applied to non-linear and non-gaussian models; a better way of parameterizing uncertainty in such situations might be to introduce Monte Carlo noise into the dataset.\cite{montecarlo}

Meanwhile, the decision was made not to estimate variances for the 304SS deconvoluted curves.  The author wishes to emphasize that this procedure was highly subjective in nature; the results depended heavily on the initial seed parameters provided to \textit{lmfit}.  The deconvolution was nevertheless an instructive exercise; the model may have utility in the future design of a rigourous fitting algorithm.

\subsection{Discussion of Hysteresis in 1018MS Curves}

As seen in Figure \ref{fig:anocat}, the 1018MS Ano/Cat and LPR curves all exhibited hysteresis between the upward and downward portions of the sweep; the downward portion consistently manifested a more-negative corrosion potential.

Recalling the discussion in the Introduction regarding the microstructure of 1018MS, one might attempt to explain this hysteresis via hydrogen reduction catalysis by the exposed Fe$_3$C lamellar microstructure.  The hystersis, however, runs in direct contradiction to the predicted effects of such a catalysis, which would enhance the exchange current density of the reduction reaction and elevate the current at potentials more negative than $\Delta \phi_{\text{corr}}$.  As a result, $\Delta \phi_{\text{corr}}$ would shift to more positive values, since the catalyzed H$^+$ reduction reaction would be able to match the rate of Fe oxidation at a lower driving force.  Yet the hystersis manifests as a shift in the opposite direction, implying that it is the exchange current of Fe oxidation that has instead been elevated.

One possible mechanism to explain this behavior would be the formation of a passivating oxide on the sample surface in between polarization sweeps.  Such a layer would depress the exchange current density of the Fe reduction reaction on the upward sweep, but once it had been ablated by vigorous Fe corrosion at positive overpotentials, it would no longer effect the exchange current density.  As such, the exchange current density on the down-sweep would appear elevated.  It is also possible that the alloying Mn in the 1018MS plays a role, though this has not been investigated.

In summary, the hysteretic behavior of the 1018MS polarization curves appears anomalous.  Further investigation is warranted to determine its causes.  The magnitude of the hysteresis, however, is consistently less than \textapprox 0.5 V; it is up to the manufacturer's discretion to determine if this is acceptable.

\subsection{Effect of Chloride Ions on 304SS Passivation Behavior}

A post-experiment inspection of the surface structure of the 304SS HCl sample demonstrated that Cl$^-$ pitting had occured, and therefore that the proposed reaction model and deconvolution were appropriate.  (See microscopy images in Appendix B.)

The maximum dissolution rate in HCl solution was found to be 8.43e-4 A/mm$^2$ at a potential of -0.157 V; these rate is 392 times larger than the maximum dissolution rate observed in H$_2$SO$_4$ solution (2.15e-6 A/mm$^2$ at -0.277 V).  It therefore appeared that the presence of chloride ions produces a marked increase in the maximum active dissolution rate, as well as shifts this maximum rate to a higher potential.  This horizontal shift may be understood as a change in the passivation behavior of 304SS in the presence of chloride ions.  This is not particularly surprising; Reference \cite{cllayer} notes that such ions are actually incorporated into the passivation layer, thereby changing its chemistry.

The similarity among the 1018MS curves for the two solutions, as well as the fact that the entire HCl polarization curve (even within the H$^+$ reduction regime) was found to be \textapprox 2 orders of magnitude higher in terms of current density compared to the H$_2$SO$_4$ curve, suggests that this may be attributed to some manner of systematic error, effects of Cl$^-$ nonwithstanding.  This set of experiments was designed simply to map the polarization curve and not to effectively quantify the difference in maximum corrosion rate; additional characterization is necessary to properly make this comparison.

The mapping and deconvolution of the 304SS HCl curve was, however, instructive in terms of other metrics.  For example, by defining the Cl$^-$ pitting potential as the point when the current contributions of the fitted Cl$^-$ component exceeded the passivation current, $\Delta \phi_{\text{pitt}}$ was found to be 0.402 V vs. SCE.  (It should be noted that this pitting current is actually a ``pseudo-current'' that has its basis in localized Fe oxidation reactions.

\subsection{Anomalous Behavior Following Passivation Potential}

There remain two anomalies within the 304SS H$_2$SO$_4$ polarization curves that must be discussed: the 9.99\% shift of the Cr III/Cr VI corrosion potential below its literature value; and the phenomenon of current depression/inversion in the region immediately following the passivation potential.

The first phenomenon may be readily explained by qualitative inspection of the deconvoluted components of the polarization curve.  As shown in Figure \ref{fig:h2so4_deconv}, the fitted component corresponding to the breakdown of water runs close to the singularity corresponding to the Cr III/Cr VI corrosion potential, providing an additional oxidative component.  This would have the effect of pushing the equilibrium potential more negative, since a greater rate of Cr VI reduction would be required to counterbalance the current produced by the breakdown of water.  Furthermore, Cr oxidation in this regime is a rate-limiting process for Fe oxidation--any Cr dissolution produces a comparable dissolution of Fe, further enhancing the manifested anodic current.

The second phenomenon, however, is not so straightforwardly explained.  The deconvolution prodedure demonstrated that it is possible to replicate this feature using components of the form of Equation \ref{eqn:bv_w} with the JMAK-type resisitivity modification of Equation \ref{eqn:rho_jmak}.  In particular, the double-singularity phenomenon at a potential of \textapprox -0.1 V in the H$_2$SO$_4$ curve may be replicated using a reduction reaction that is heavily ``passivated'' at more negative potentials.  Oxygen reduction was categorically disqualified as a candidate; in such solutions as 1M HCl and 1M H$_2$SO$_4$, oxygen gas has a low solubility,\cite{labguide} while an inspection of the electrode surface near the solution/air interface did not yield evidence of any increased corrsion.  Chemical species from the counter-electrode were also disqualified; the Pt counter-electrode was specifically chosen such that it would not corrode into the solutions.\cite{devine}

Instead, the behavior of the curve suggests that this effect comes from the reduction of an aqueous species that was generated by the corrosion of the sample itself (i.e. the reduction of ions of Fe, Ni, Cr, or other alloying elements).  The procedure did not provide a rigorous means of identifying the involved species; in particular, the fitting behavior in this region is heavily dependent upon the curvature of the fitted Fe passivation component.  Once again, a more rigourous analysis would be required in order to properly characterize this region, but the deconvolution yields an important conclusion: this is a transient effect based upon a modified solution chemistry, and as such, the manufacturer should not construe it as improved passivative properties in its 304SS samples.

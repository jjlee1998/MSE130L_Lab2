- 1/8" diameter rods of 304SS or 1018MS were cut into cylindrical samples, then sanded with 600-grit paper and rinsed with deionized water so as to slean their surfaces.  Four samples were produced in total (2x 304SS and 2x 1018MS).  The samples were then imaged under an optical microscope so as to provide a reference against which to compare the corroded surfaces.

- Each sample was installed as the working electrode within a polarization cell, alongside a platinum counter-electrode and a standard calomel reference electrode (SCE).  So as to minimize the effects of solution resistivity on the measured voltage, the reference electrode was contained within a Lugger-Habin probe, of which the capillary tip was placed at the midpoint of the sample.  The polarization cell was filled with either 250 mL of 1M HCl solution or 250 mL of 1M H2SO4 solution.  The Luggin-Haber probe was filled with the same chosen solution to a surface level 1/4" below that of the polarization cell; this precluded the contamination of the polarization cell with 4M Cl- solution from the reference electrode.  Care was taken during filling to avoid bubbles, as well as to position the Luggin-Haber probe in a way that would not allow bubbles to enter its capillary during the course of the experiment.  Sample dimensions, solutions, and immersion lengths are recorded in Table 1.

- TABLE N (sample diameter, immersion lengths)

- The three electrodes were connected to a potentiostat constructed by previous laboratory personnel (13.5V compliance voltage, 190 mA maximum current, 200 mA thermal fuse cutoff, 1e-6 mA current sensitivity).  The approximate location of the H/Fe corrosion potential was identified, after which various polarization sweeps were conducted for each sample.  Twelve polarization sweeps were collected in total.  The parameters of each sweep are enumerated in Table 2.  After the sweeps were concluded, the samples were dried and imaged once more under an optical microscope at magnifications of 5x, 10x, and 20x.

Scan # - Sample - Solution - Sweep Classifaction - Sweep Rate - Description

1 - 1018MS - H2SO4 - Anodic/Cathodic 1 mA - 1 mV/sec - Upward scan from 200 mV below CP to a potential above CP which produces 1 mA anodic current, followed by a reverse scan to return to the starting potential (200 mV below CP).
2 - 1018MS - H2SO4 - LPR 1 mA - 0.1 mV/sec - After Scan 1, potential held at 200 mV until stable current, followed by an upward scan from 20 mV below CP to 20 mV above CP and a downward scan to return to 20 mV below CP.
3 - 1018MS - H2SO4 - Anodic/Cathodic 10 mA - 1 mV/sec - Upward scan from 200 mV below CP to a potential above CP which produces 10 mA anodic current, followed by a reverse scan to return to the starting potential (200 mV below CP).
4 - 1018MS - H2SO4 - LPR 10 mA - 0.1 mV/sec - After Scan 3, potential held at 200 mV until stable current, followed by an upward scan from 20 mV below CP to 20 mV above CP and a downward scan to return to 20 mV below CP.
5 - 1018MS - HCl - Anodic/Cathodic 1 mA - 1 mV/sec - Upward scan from 200 mV below CP to a potential above CP which produces 1 mA anodic current, followed by a reverse scan to return to the starting potential (200 mV below CP).
6 - 1018MS - HCl - LPR 1 mA - 0.1 mV/sec - After Scan 5, potential held at 200 mV until stable current, followed by an upward scan from 20 mV below CP to 20 mV above CP and a downward scan to return to 20 mV below CP.
7 - 1018MS - HCl - Anodic/Cathodic 10 mA - 1 mV/sec - Upward scan from 200 mV below CP to a potential above CP which produces 10 mA anodic current, followed by a reverse scan to return to the starting potential (200 mV below CP).
8 - 1018MS - HCl - LPR 10 mA - 0.1 mV/sec - After Scan 7, potential held at 200 mV until stable current, followed by an upward scan from 20 mV below CP to 20 mV above CP and a downward scan to return to 20 mV below CP.
9 - 304SS - H2SO4 - Cathodic - 1 mV/sec - Downward scan from 20 mV above CP to ~700 mV below CP.
10 - 304SS - H2SO4 - Anodic - 1 mV/sec - Upward scan from 20 mV below CP to the potential (~1700 mV vs. SCE) that achieves maximum potentiostat current (190 mA), followed by a downward scan until the sample surface changes from anodic to cathodic (i.e. current sign flips).
11 - 304SS - HCl - Cathodic - 1 mV/sec - Downward scan from 20 mV above CP to ~700 mV below CP.
12 - 304SS - HCl - Anodic - 1 mV/sec - Upward scan from 20 mV below CP to the potential (~400 mV vs. SCE) that achieves maximum potentiostat current, followed by a downward scan until the sample surface changes from anodic to cathodic (i.e. current sign flips).  Since the current in this sample/solution combination may continue to increase after switching scan direction, the direction was reversed in the vicinity of 100 mA instead of the maximum current (190 mA).

